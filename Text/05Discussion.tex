\documentclass[../00main.tex]{subfiles}
\graphicspath{{../}}

\begin{document}

\section{The modelled wavefield}

From the goodness-of-fit analysis and visual comparison of spectra and waveforms in Section \ref{sec:validation}, it becomes clear that the 3D model can predict the peak ground values well when it is compared to observed data. Peak-based values scored in the 'good' and 'excellent' category for both the simulation at a period of 15 s and the simulations at 1 s. The fit for these peak-based values improves with epicentral distance for the 3D model. The 1D model performs less good, with average scores mainly in the 'fair' category for the PGV at 15 s. Duration is not captured accordingly by the model, which affected the scores for the Arias intensity and energy integral score. The match of the waveform decreases when increasing the maximum frequency to 1 Hz, but the amplitudes remained well in the same range. This makes the peak-based parameters such as PGV suitable ground motion proxies to study using this model. Possibly, the absence of a detailed model of the shallow crust can improve the performance regarding duration and spectral content of the signal. \citep{imperatori2015role} recommend usage of a detailed shallow subsurface model. Duration can be significantly affected by the presence of thin sedimentary layering, creating backscattering and trapping waves. When approaching higher frequencies, smaller-scale features become more relevant and a more adequate estimate can be made for i.e. building design. In building design, the duration and spectral content of the earthquake play an important role.


\section{The influence of the model, topography and an ocean layer}

The initial reference patterns that come forward when imaging the PGV are heavily influenced by the source mechanism. Close to the source, the direct imprint of the wavefield pattern caused by the point source is visible. When compared to the other simulations using a 3D subsurface model, the 1D model over-estimates this signal mainly in the fault-normal and parallel components containing the shear components of the source, and it under-estimates the peak velocities for the oblique components containing mainly the compressional energy. The 3D model still has this strong source imprint in the pattern, but shows a more perturbed pattern, especially at radial distances > 20 km. In the far field, beyond this distance, the effect of the 3D model is better visible. As shown by \hl{cite}, usage of a 3D model is highly influential on the wavefield pattern and should always be used if available. The scale of the domain used here is limited and the 3D model is accurate up to a period of 8 s, so in the range of 800-300 m of resolution, based on velocities ranging from roughly 6000 at depth to 2400 $ms^{-1}$ at the surface. This means that at the scale of the domain considered (100 $\times$ 80 km), the variations remain limited. When considering a larger domain this effect may be even more considerable. Addition of topography has amplified all off-diagonal components but the north-west facing one, in the case of CMT1. The steep ridge that is located to the north-west of the source does not only deform the source pattern as observed in the two flat domains, but it also de-amplifies mainly the compressive wavefield directed normal to the ridge. However on top of the ridge, further towards the north-west, the signal is again amplified. Back-scattering of the wavefield near a steep increase in topography is a possible cause \hl{cite}. Receiver MIAN, located on an island exactly above this steep ridge, reports significant amplification in the E-component with respect to the flat domains. This effect was also visible in the vertical component and less pronounced in the fault-normal north component. The compressional energy in the form of P-waves and Rayleigh surface waves are more affected by the topographical effects than the shear component. Finally, inclusion of the ocean layer in the domain has a pronounced imprint on the pattern of the PGV. This imprint coincides with both a steep incline in topography, as well as the transition from a solid-fluid coupled ocean formulation to a ocean loading formulation. The ocean formulation still influence the results on an island \hl{reference}. Receiver OCWE was additionally chosen from the receiver grid that was used in the simulation, and is located on the ocean bottom in the deeper part of the Marmara sea \hl{how deep?}. Both stations measure a significant effect that is mainly seen in the surface wave coda. For MIAN, this effect was only pronounced in the vertical component. Furthermore, the increased amplitudes caused by the addition of surface topography are attenuated by the ocean layer here. Station OCWE sees this effect in all three components. For all four evaluated stations, the first arriving P- and S-wave phases are similar, except for the 1D model result. The main differences start to arise in the surface wave and coda of the wave, which are also the main points of interest in terms of peak ground motion and seismic hazard \hl{}. Even though the differences caused in the patterns as seen here may remain similar, the variations are soon within the range of tens of kilometres spatially, and with factors of difference in maximum intensity above 1.4 between the used models. 

 
\section{Strike, dip and rake variation}

Additionally, variation of the source mechanism itself alters these patterns as described. The source components are subject to uncertainty coming forward from the inversion for the source, which yields different results depending on the wave types and coverage of an event, as well as the model used for inversion \hl{cite}. Variation of the fault orientation, in this case the strike, results in rotation of this pattern. In case of a pure strike-slip fault such as CMT1, the axis of rotation is the vertical axis, and the pattern turns clockwise or anti-clockwise with respect to the north. The main differences with respect to CMT1 accumulate here along the initial fault strike for the east component and along the off-diagonals for the north components. The vertical component sees a large difference mainly along the initial fault strike. With progressing strike alteration, the largest differences are found in these accumulation regions. Depending on the initial strike, the addition of a 3D model, topography and ocean changes the amount of variation in PGV caused by the strike changes. For the horizontal axes this depends on the orientation of the fault, and for the vertical axis the variation pattern is always perturbed and decreased in intensity when adding a 3D model, topography and the ocean layer. Changing the dip of the fault, whilst leaving the rake and strike unchanged, shows the same trend: the maximum change in PGV is accumulated in certain regions for the horizontal and vertical components, dependent on the initial fault orientation. Again, variation in the horizontal component is de-amplified for the fault-normal and parallel radiation pattern changes, whereas it is amplified in the orientations oblique from the initial fault strike. 

So for progressive inclusion of more complexities to the domain, starting with the 3D model, topography and an ocean layer, the vertical component shows a decrease in the maximum variation of PGV measured in the domain, as well as the orientations normal to the fault. 

The maximum PGV recorded in the domain follows very distinct trends when it comes to the variation of strike, dip and rake. These trends are closely related to the progression of the individual moment tensor components when the strike, dip and rake are varied. It is more straightforward to study these patterns in a case where the initial orientation of the fault is in terms of $\frac{pi}{2}$, so that the effect of the individual affected components can be studied. The trends are periodic and symmetrical for the 1D model, which contains no lateral variations, but show increased perturbations and asymmetry when including complexities such as the 3D model, topography and ocean coupling. 



\subsection{Depth of the source}




\subsection{Implications of the results: the bigger picture}

Of course it is dependent on the subsurface model, the amount and shape of the topography and possibly the presence of an ocean in a model domain, and thus location dependent where exactly the zones of accumulation of maximum PGV change occur. Therefore, the amount of attenuation or amplification of changes in source orientation is also location dependent when including these complexities. This means that the results obtained here can not be directly extrapolated to any region or source type. The fault in Istanbul is located in a basin close to a steep ridge. Topographic effects may be very different when it would be located not in a basin but on a surface elevation with respect to the surrounding area. 

In the case of this study near Istanbul, the fault in question is the Princes' Island segment of the Marmara segment of the NAFZ. Because of the imminent large seismic hazard, this segment has been very well studied. This means that the ranges of strike, dip and rake can be constrained by tectonic and geologic knowledge of the fault. In case of the strike, dip and rake of the source, using the separate simulation of the six moment tensor components as executed in this study, requires only six simulations. With these six simulations, the recorded surface pattern of any combination of moment tensor components can be computed. However separate simulation is needed for changes in the depth or location (not considered in this study) of the source. 

As discussed in the introduction of this thesis, the rapid CMT characterisation is the precursor step of a realistic large rupture simulation. The vast computational costs of these rupture simulations (\hl{cite}) compared to a point source simulation asks for more detailed study of the earthquake rupture, and thus the parameters and uncertainties of the rupture and orientation will be known into some more detail. It is beneficial to directly compute the patterns of variation within the boundary of these uncertainty estimates. \hl{cite MRA} works with the \textit{maximum rotated angle} (MRA) of the CMT, which is a measure of the variation of strike, dip and rake of all CMT mechanisms within the considered possible range.  

Simulations in this study were conducted using the 3D model by \citep{cubuk-sabuncu_3-d_2017}. No model for the shallow subsurface was used. Possible extension of this studies whereby a detailed structure of the upper crust is included can greatly enhance the realistic surface wave patterns. Path- and site effects studied by \hl{cite} could cause amplifications of ground motions up to 50\%. This does not mean that the results of the current study are incorrect. They form the literal basic structure of the pattern that is then affected by the upper layer of the crust. 







% iets over de MRA, dat het interessant zou zijn om de dichtsbijzijnde opties van MRA uit te proberen als uncertainty range, in 1x strike, dip en rake veranderen.





\end{document}


